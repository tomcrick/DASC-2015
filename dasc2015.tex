\documentclass[conference]{IEEEtran}

\usepackage[british]{babel}
\usepackage{graphicx}
\usepackage[hyphens]{url}
\usepackage{enumerate}
\usepackage[noadjust]{cite}
\usepackage[pdftex,colorlinks=true]{hyperref}


% correct bad hyphenation here
%\hyphenation{op-tical net-works semi-conduc-tor}


\begin{document}
%
% paper title
\title{CCTV as Smart Sensor Networks}


% author names and affiliations
% use a multiple column layout for up to three different
% affiliations
\author{
    \IEEEauthorblockN{Giles Oatley, Tom Crick, Dee Bolt}
    \IEEEauthorblockA{Department of Computing,
      Cardiff Metropolitan University, UK
    \\\{goatley,tcrick,dbolt\}@cardiffmet.ac.uk}
}

% conference papers do not typically use \thanks and this command
% is locked out in conference mode. If really needed, such as for
% the acknowledgment of grants, issue a \IEEEoverridecommandlockouts
% after \documentclass


% use for special paper notices
%\IEEEspecialpapernotice{(Invited Paper)}


% make the title area
\maketitle


\begin{abstract}
With the emergence of so-called ``smart CCTV'' being able to recognise
the precursors for disorder and civil disobedience, we present a study
into using available CCTV networks augmented with social media
datasets.

We examine the existing CCTV infrastructure in the UK, and use an
agent-based simulation to model interactions between people based on
friendship networks and features derived from their social media
usage, proposing a novel algorithm for detection of
psychopathy. Finally, we explore the frequency of crimes occurring
within CCTV viewsheds using available UK police crime datasets to
illustrate the current limitations of the CCTV infrastructure.
\end{abstract}

% For peer review papers, you can put extra information on the cover
% page as needed:
% \ifCLASSOPTIONpeerreview
% \begin{center} \bfseries Keywords here... \end{center}
% \fi
%
% For peerreview papers, this IEEEtran command inserts a page break and
% creates the second title. It will be ignored for other modes.
%\IEEEpeerreviewmaketitle

\begin{IEEEkeywords}
CCTV, Smart Cities, Sensors, Networks, Crowd Behaviour, Traits, Agent-Based
Modelling, Social Media
\end{IEEEkeywords}


\section{Introduction}

We are interested in exploring how closed-circuit television (CCTV)
can be combined with analysis of large-scale social media datasets to
determine the general ``mood'' of a crowd, and to explore the
potential and limitations of this hybrid approach to behaviour
modelling. To this end we present a background to the CCTV
infrastructure (and policy) in the UK, including the numbers and
quality of the cameras networks involved. We also provide a review of
the latest research deriving behaviours and traits from social media
datasets.

The proliferation of CCTV camera networks across urban communities in
the UK has received a mixed reception. There has been significant
criticism as local authorities have spent \pounds515m on CCTV and associated
infrastructure in four years ~\cite{bbw:2012}, with mixed
results. Furthermore, they are often reluctant to reveal how effective
they have been for monitoring of crime and anti-social behaviour. 
Nevertheless, with the increased focus on the development (and
associated e-infrastructure) of smart cities, there is often
significant funding available for ``smart'' CCTV systems, claiming to
prevent crime by detecting crowd characteristics indicative of
criminal behavior ~\cite{welsh+farrington:2009,dibella-et-al:2014}.


% Cite papers: HAS2014~\cite{oatley+crick:2014}, ASONAM
% 2014~\cite{oatley+crick_asonam2014}, FOSINT-SI
% 2014~\cite{oatley+crick_fosintsi2014}, SNAM~\cite{oatley+crick:2015}, etc.

%\newpage

% trigger a \newpage just before the given reference
% number - used to balance the columns on the last page
% adjust value as needed - may need to be readjusted if
% the document is modified later
%\IEEEtriggeratref{37}
% The "triggered" command can be changed if desired:
%\IEEEtriggercmd{\enlargethispage{-5in}}

% references section
\bibliographystyle{IEEEtran}
\bibliography{dasc2015}

% that's all folks
\end{document}


