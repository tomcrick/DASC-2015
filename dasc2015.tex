\documentclass[conference]{IEEEtran}

\usepackage[british]{babel}
\usepackage{graphicx}
\usepackage[hyphens]{url}
\usepackage{enumerate}
\usepackage[noadjust]{cite}
\usepackage[pdftex,colorlinks=true]{hyperref}


% correct bad hyphenation here
%\hyphenation{op-tical net-works semi-conduc-tor}


\begin{document}
%
% paper title
\title{CCTV as Smart Sensor Networks}


% author names and affiliations
% use a multiple column layout for up to three different
% affiliations
\author{
    \IEEEauthorblockN{Giles Oatley, Tom Crick, Dee Bolt}
    \IEEEauthorblockA{Department of Computing,
      Cardiff Metropolitan University, UK
    \\\{goatley,tcrick,dbolt\}@cardiffmet.ac.uk}
}

% conference papers do not typically use \thanks and this command
% is locked out in conference mode. If really needed, such as for
% the acknowledgment of grants, issue a \IEEEoverridecommandlockouts
% after \documentclass


% use for special paper notices
%\IEEEspecialpapernotice{(Invited Paper)}


% make the title area
\maketitle


\begin{abstract}
With the emergence of so-called ``smart CCTV'' being able to recognise
the precursors for disorder and civil disobedience, we present a study
into using available CCTV networks augmented with social media
datasets.

We examine the existing CCTV infrastructure in the UK, and use an
agent-based simulation to model interactions between people based on
friendship networks and features derived from their social media
usage, proposing a novel algorithm for detection of
psychopathy. Finally, we explore the frequency of crimes occurring
within CCTV viewsheds using available UK police crime datasets to
illustrate the current limitations of the CCTV infrastructure.
\end{abstract}

% For peer review papers, you can put extra information on the cover
% page as needed:
% \ifCLASSOPTIONpeerreview
% \begin{center} \bfseries Keywords here... \end{center}
% \fi
%
% For peerreview papers, this IEEEtran command inserts a page break and
% creates the second title. It will be ignored for other modes.
%\IEEEpeerreviewmaketitle

\begin{IEEEkeywords}
CCTV, Smart Cities, Sensors, Networks, Crowd Behaviour, Traits, Agent-Based
Modelling, Social Media
\end{IEEEkeywords}


\section{Introduction}

We are interested in exploring how closed-circuit television (CCTV)
can be combined with analysis of large-scale social media datasets to
determine the general ``mood'' of a crowd, and to explore the
potential and limitations of this hybrid approach to behaviour
modelling. To this end we present a background to the CCTV
infrastructure (and policy) in the UK, including the numbers and
quality of the cameras networks involved. We also provide a review of
the latest research deriving behaviours and traits from social media
datasets.

The proliferation of CCTV camera networks across urban communities in
the UK has received a mixed reception. There has been significant
criticism as local authorities have spent \pounds515m on CCTV and
associated infrastructure in four years ~\cite{bbw:2012}, with mixed
results. Furthermore, they are often reluctant to reveal how effective
they have been for monitoring of crime and anti-social
behaviour\footnote{For example:
\url{http://getthedata.org/questions/158/locations-of-council-operated-cctv-cameras-in-the-uk/}}.
Nevertheless, with the increased focus on the development (and
associated e-infrastructure) of smart cities, there is often
significant funding available for ``smart'' CCTV systems, claiming to
prevent crime by detecting crowd characteristics indicative of
criminal behavior ~\cite{welsh+farrington:2009,dibella-et-al:2014}.

CCTV is currently used as a deterrent in the UK and many other
countries; in terms of crime prevention, an examination of crime in
the viewshed of publicly funded CCTV cameras in Philadelphia, USA,
found that the introduction of cameras was associated with a 13\%
reduction in crime~\cite{ratcliffe+taniguchi:2008}. Research found
that while there appears to be a general benefit to the cameras, there
were as many sites that showed no benefit of camera presence as there
were locations with a positive outcome on
crime~\cite{ratcliffe-et-al:2009}.  A further study in Newark, USA,
found that strategically-placed cameras were not any different from
randomly-placed cameras at deterring crime within their viewsheds
although there were significant improvements to location quotient
values for gun shootings and automobile thefts after camera
installations~\cite{caplan-et-al:2011}.

However, we are interested in CCTV as a sensor, for validation of
models derived from sources such as social media
analysis. Over-reliance on sophisticated software products such
`intelligent network products' and geographical techniques such as
`hotspot analysis' can lead to weak critical thinking. Thus, the
next-generation (social network) analysis must focus much more
intensely on the content of the contacts, on the social context, and
on the interpretation of such information~\cite{klerks:2001}.

{\emph{Analysis of Competing Hypotheses}}
(ACH\footnote{\url{http://www2.parc.com/istl/projects/ach/ach.html}})
is an important software tool used for intelligence analysis, and can
ameliorate human operators misinterpreting results and preventing
cognitive biases in inferences which may contaminate the process and
ultimately the decision reached. Numerous data sources and techniques
such as statistical profiling are available to the analyst, however we
can often exhibit a ``confirmatory
bias''~\cite{lipton:2002,lipton:2005} in that we focus upon data which
conforms to the initial ideas formed (e.g. about who is the suspect)
and so fail to test them by seeking evidence which contradicts our
notions. Other errors of judgment include counterfactual thinking,
illusory correlation, false consensus bias, ignoring base rate
information, culture/gender biases, group effects and so on.

\section{Social Media and Personality}\label{socmed+pers}

The work of Schwartz et al.~\cite{schwartz-et-al:2013} analysed what
people say in social media to find distinctive words, phrases, and
topics as functions of known attributes of people such as gender, age,
location, or psychological characteristics. This can thus be
transposed, inferring gender, age and so on, from social media
data. The negative implications of these developments are that they
can easily be applied to large numbers of people without obtaining
their individual consent or even being aware. Commercial companies,
governmental institutions, or even one’s Facebook friends could use
software to infer personality (and other attributes, such as
intelligence or sexual orientation) that an individual may not have
intended to share~\cite{lambiotte+kosinski:2014}.

There are now numerous categories of social media sites; for example:
social networking sites (e.g. Facebook), professional networking
(e.g. LinkedIn), microblogging (e.g. Twitter, Tumblr), wiki-based
knowledge sharing sites (e.g.  Wikipedia), social news/websites of
news media (e.g. Huffington Post), forums, mailing lists, newsgroups,
community media sites (e.g. YouTube, Flickr, Instagram), social Q\&A
sites (e.g. Quora, Yahoo Answers), user reviews (e.g. Yelp, Amazon,
TripAdvisor), social curation sites (e.g., Reddit, Slashdot,
Pinterest), location-based social networks (e.g., Foursquare), etc.

Caverlee et al.~\cite{caverlee-et-al:2013} discuss the knowledge
discovery and data mining stages applied to geospatial data found in
social media, developing geo-social intelligence. They describe the
geo-social overlay of the physical environment of the planet; consider
for example, the overlay of the six billion geotagged
tweets\footnote{\url{https://www.mapbox.com/blog/twitter-map-every-tweet}}. Both
they and Stefanidis et al.~\cite{stefanidis-et-al:2014} discuss the
need for new systems and techniques to leverage these footprints, as
well as the wide range of challenges and opportunities to the
geospatial intelligence community in particular.



% Cite papers: HAS2014~\cite{oatley+crick:2014}, ASONAM
% 2014~\cite{oatley+crick_asonam2014}, FOSINT-SI
% 2014~\cite{oatley+crick_fosintsi2014}, SNAM~\cite{oatley+crick:2015}, etc.

%\newpage

% trigger a \newpage just before the given reference
% number - used to balance the columns on the last page
% adjust value as needed - may need to be readjusted if
% the document is modified later
%\IEEEtriggeratref{37}
% The "triggered" command can be changed if desired:
%\IEEEtriggercmd{\enlargethispage{-5in}}

% references section
\bibliographystyle{IEEEtran}
\bibliography{dasc2015}

% that's all folks
\end{document}


